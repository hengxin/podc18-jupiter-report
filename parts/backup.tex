% file: parts/backup.tex

\appendix

%%%%%%%%%%%%%%%%%%%%
\begin{frame}{}
  \centerline{\teal{\Huge Backup}}
\end{frame}
%%%%%%%%%%%%%%%%%%%%

%%%%%%%%%%%%%%%%%%%%
\begin{frame}{}
  \centerline{\Large \teal{Replication {\normalsize (for availability)}}}
  \vspace{-0.20cm}
  
  \fignocaption{width = 0.50\textwidth}{figs/googledoc-replication}

  \pause
  \vspace{-0.50cm}
  \begin{center}
    Replicas respond to user operations \red{immediately}  \\[3pt]

    Updates are propagated \red{asynchronously}
  \end{center}
\end{frame}
%%%%%%%%%%%%%%%%%%%%

%%%%%%%%%%%%%%%%%%%%
\begin{frame}{}
  \centerline{It is still challenging to achieve convergence despite the server.}
  
  \fignocaption{width = 0.50\textwidth}{figs/jupiter-cs}

  \begin{center}
    Serializability may not be desirable. \\[6pt]
    It does not imply that clients process operations in the same order.
  \end{center}
\end{frame}
%%%%%%%%%%%%%%%%%%%%

%%%%%%%%%%%%%%%%%%%%
\begin{frame}{}
  \[
    \forall \sigma, \sigma': a, b \in \sigma \cap \sigma' \implies (a \prec_{\sigma} b \iff a \prec_{\sigma'} b)
  \]
  \[
    (\sigma, \sigma': \text{ list}; \quad a, b: \text{ element}; \quad \prec_{\sigma}: \text{ precedes})
  \]

  \begin{columns}
    \column{0.50\textwidth}
      \fignocaption{width = 0.60\textwidth}{figs/ex-weak-list-spec}
      \vspace{-0.60cm}
      \fignocaption{width = 0.40\textwidth}{figs/red-cross}
    \column{0.50\textwidth}
      \fignocaption{width = 0.75\textwidth}{figs/ex-strong-list-spec}
      \vspace{-0.60cm}
      \fignocaption{width = 0.30\textwidth}{figs/green-check}
  \end{columns}
\end{frame}
%%%%%%%%%%%%%%%%%%%%

%%%%%%%%%%%%%%%%%%%%
\begin{frame}{}
  \centerline{OT (Operational Transformation)~\ncite{Ellis:SIGMOD89}}

  \begin{columns}
    \column{0.40\textwidth}
      \begin{center}
	% file: tikz/no-ot-tcs06.tex

\newcommand{\ins}[2]{\textsc{Ins}(#1,#2)}
\newcommand{\del}[2]{\textsc{Del}(#2)}  % del(a, p): ignoring the deleted element ``a''

\begin{tikzpicture}[
	timeline/.style = {thick, dashed}, 
	>=Stealth, 
	send/.style = {>=Stealth, ->},
	list/.style = {rectangle, draw, inner sep = 5pt, outer sep = 2pt, fill = #1, font = \large},
	op/.style = {font = \footnotesize, #1}
  ]
  \uncover<2->{
    \node[list = blue!20, label = {above:{$R_1$}}] (r1) {efecte}; 
    \node[list = blue!20, right = 1.20cm of r1, label = {above:{$R_2$}}] (r2) {efecte};

    \foreach \r/\rbot in {r1/r1bot, r2/r2bot} {
      \node[below = 4.0cm of \r] (\rbot) {};
    }

    \draw[timeline] ($(r1.south)+(0,-5pt)$) -- ($(r1bot.south)+(0,-15pt)$);
    \draw[timeline] ($(r2.south)+(0,-5pt)$) -- ($(r2bot.south)+(0,-15pt)$);
  }

  \uncover<3->{
    \node (ins) [op = purple] at ($(r1)!0.25!(r1bot)$) {$o_1 = \ins{f}{1}$};
    \node (del) [op = purple] at ($(r2)!0.25!(r2bot)$) {$o_2 = \del{e}{5}$};

    \node (r11) [list = teal!20] at ($(r1)!0.50!(r1bot)$) {effecte};
    \node (r21) [list = teal!20] at ($(r2)!0.50!(r2bot)$) {efect};
  }

  \uncover<4->{
    \node (ins') [op] at ($(r2)!0.75!(r2bot)$) {$o_1' = \ins{f}{1}$};
    \node (del') [op] at ($(r1)!0.75!(r1bot)$) {$o_2' = \del{e}{5}$};

    \draw[send] (ins) -- (ins');
    \draw[send] (del) -- (del');

    \node (r12) [list = red!20] at ($(r1)!1.00!(r1bot)$) {effece};
    \node (r22) [list = red!20] at ($(r2)!1.00!(r2bot)$) {effect};
  }
\end{tikzpicture}
      \end{center}
    \column{0.40\textwidth}
      \begin{center}
	% file: tikz/ot-tcs06.tex

\newcommand{\ins}[2]{\textsc{Ins}(#1,#2)}
\newcommand{\del}[2]{\textsc{Del}(#2)}  % del(a, p): ignoring the deleted element ``a''

\begin{tikzpicture}[
	timeline/.style = {thick, dashed}, 
	>=Stealth, 
	send/.style = {>=Stealth, ->},
	list/.style = {rectangle, draw, inner sep = 5pt, outer sep = 2pt, fill = #1, font = \large},
	op/.style = {font = \footnotesize, #1}
  ]

  \uncover<5->{
    \node[list = blue!20, label = {above:{$R_1$}}] (r1) {efecte}; 
    \node[list = blue!20, right = 1.20cm of r1, label = {above:{$R_2$}}] (r2) {efecte};

    \foreach \r/\rbot in {r1/r1bot, r2/r2bot} {
      \node[below = 4.0cm of \r] (\rbot) {};
    }

    \draw[timeline] ($(r1.south)+(0,-5pt)$) -- ($(r1bot.south)+(0,-15pt)$);
    \draw[timeline] ($(r2.south)+(0,-5pt)$) -- ($(r2bot.south)+(0,-15pt)$);

    \node (ins) [op = purple] at ($(r1)!0.25!(r1bot)$) {$o_1 = \ins{f}{1}$};
    \node (del) [op = purple] at ($(r2)!0.25!(r2bot)$) {$o_2 = \del{e}{5}$};

    \node (r11) [list = teal!20] at ($(r1)!0.50!(r1bot)$) {effecte};
    \node (r21) [list = teal!20] at ($(r2)!0.50!(r2bot)$) {efect};
  }

  \uncover<6->{
    \node (ins') [op] at ($(r2)!0.75!(r2bot)$) {$o_1' = \ins{f}{1}$};
    \node (del') [op, red, ellipse, draw] at ($(r1)!0.75!(r1bot)$) {$o_2' = \del{e}{6}$};

    \draw[send] (ins) -- (ins');
    \draw[send] (del) -- (del');

    \node (r12) [list = green!20] at ($(r1)!1.00!(r1bot)$) {effect};
    \node (r22) [list = green!20] at ($(r2)!1.00!(r2bot)$) {effect};
  }
\end{tikzpicture}

      \end{center}
  \end{columns}
\end{frame}
%%%%%%%%%%%%%%%%%%%%

%%%%%%%%%%%%%%%%%%%%
\begin{frame}{}
  \centerline{\large OT functions for a replicated list object~\ncite{Ellis:SIGMOD89}}

  \resizebox{\textwidth}{!}{
    \begin{minipage}{\textwidth}
      % file: parts/list-ot.tex

\newcommand{\ldel}{\textsc{Del}}	% List DEL
\newcommand{\lins}{\textsc{Ins}}	% List INS
\newcommand{\ldelop}[2]{\ldel\left(#1, #2\right)}  % #1: index, #2: priority
\newcommand{\linsop}[3]{\lins\left(#1, #2, #3\right)}  % #1: index, #2: element, #3: priority

%%%%%%%%%% list-ot.tex %%%%%%%%%% 
\begin{align*}
  % ins vs. ins
  OT\Big(\lins(a_1, p_1, pr_1), \lins(a_2, p_2, pr_2)\Big) &= \begin{cases}
    \lins(a_1, p_1, pr_1) 		& p_1 < p_2 \\[3pt]
    \lins(a_1, p_1 + 1, pr_1) 		& p_1 > p_2 \\[3pt]
    \textsc{NOP} 			& p_1 = p_2 \land a_1 = a_2 \\[3pt]
    \lins(a_1, p_1 + 1, pr_1) 		& p_1 = p_2 \land a_1 \neq a_2 \land pr_1 > pr_2 \\[3pt]
    \lins(a_1, p_1, pr_1)		& p_1 = p_2 \land a_1 \neq a_2 \land pr_1 \le pr_2
  \end{cases} \\[10pt]
  % ins vs. del
  OT\Big(\lins(a_1, p_1, pr_1), \ldel(\_, p_2, pr_2)\Big) &= \begin{cases}
    \lins(a_1, p_1, pr_1) 		& p_1 \le p_2 \\[3pt]
    \lins(a_1, p_1 - 1, pr_1) 		& p_1 > p_2
  \end{cases} \\[10pt]
  % del vs. ins
  OT\Big(\ldel(\_, p_1, pr_1), \lins(a_2, p_2, pr_2)\Big) &= \begin{cases}
    \ldel(\_, p_1, pr_1) 		& p_1 < p_2 \\[3pt]
    \ldel(\_, p_1 + 1, pr_1) 		& p_1 \ge p_2
  \end{cases} \\[10pt]
  % del vs. del
  OT\Big(\ldel(\_, p_1, pr_1), \ldel(\_, p_2, pr_2)\Big) &= \begin{cases}
    \ldel(\_, p_1, pr_1) 		& p_1 < p_2 \\[3pt]
    \ldel(\_, p_1 - 1, pr_1) 		& p_1 > p_2 \\[3pt]
    \textsc{NOP}			& p_1 = p_2
  \end{cases} \\
\end{align*}
%%%%%%%%%% list-ot.tex %%%%%%%%%% 

    \end{minipage}
  }
\end{frame}
%%%%%%%%%%%%%%%%%%%%

%%%%%%%%%%%%%%%%%%%%
\begin{frame}{}
  \centerline{\large Consider a replicated system with $n \; (=3)$ clients.}

  \fignocaption{width = 0.50\textwidth}{figs/jupiter-schedule}
\end{frame}
%%%%%%%%%%%%%%%%%%%%

%%%%%%%%%%%%%%%%%%%%
\begin{frame}{}
  \begin{Theorem}[Equivalence of CJupiter and Jupiter]
    Under the same schedule, the behaviors of corresponding replicas in CJupiter and Jupiter are the same.
  \end{Theorem}

  \vspace{0.30cm}
  \centerline{\large At the server side:}
  \begin{prop}[$n \leftrightarrow 1$ (Informal)]
    The single $n$-ary ordered state space at the server side in CJupiter 
    is a \red{compact representation} of $n$ $2D$ state spaces at the server side in Jupiter.
  \end{prop}

  \vspace{0.30cm}
  \centerline{\large At the client side:}
  \begin{prop}[$1 \leftrightarrow 1$ (Informal)]
    Jupiter is \red{slightly optimized in implementation} at clients by eliminating redundant OTs than CJupiter.
  \end{prop}
\end{frame}
%%%%%%%%%%%%%%%%%%%%
