% file: parts/jupiter.tex

%%%%%%%%%%%%%%%%%%%%
\begin{frame}{}
  \centerline{\Huge \teal{Jupiter}}
\end{frame}
%%%%%%%%%%%%%%%%%%%%

%%%%%%%%%%%%%%%%%%%%
\begin{frame}{}
  \centerline{\large Jupiter adopts the \red{client-server} architecture~\ncite{Nichols:UIST95}:}

  \begin{center}
    \begin{minipage}{0.50\textwidth}
      \input{tikz/jupiter-cs-tikz}
    \end{minipage}
  \end{center}

  \vspace{-1.00cm}
  \uncover<2->{
    \begin{center}
      Operations are \red{totally ordered} at the server. \\[6pt]
      Client $\xrightarrow[]{\quad \text{FIFO} \quad}$ server $\xrightarrow[]{\quad \text{FIFO} \quad}$ other clients
    \end{center}
  }
\end{frame}
%%%%%%%%%%%%%%%%%%%%

%%%%%%%%%%%%%%%%%%%%
\begin{frame}{}
  \begin{center}
    {\large To achieve convergence, Jupiter uses \red{$2D$ state spaces}~\ncite{Xu:CSCW14}
    to manage how and when to perform \red{OTs}~\footnote{OT: Operational Transformation}~\ncite{Ellis:SIGMOD89}.}
  \end{center}

  \fignocaption{width = 0.32\textwidth}{figs/2d-statespace}

  \begin{center} 
    There can be \red{$\le 2$ edges} coming from the same node ({\textsc{\footnotesize Local}} or \textsc{\footnotesize Global}).
  \end{center}
\end{frame}
%%%%%%%%%%%%%%%%%%%%

%%%%%%%%%%%%%%%%%%%%
\begin{frame}{}
  \centerline{\large Each \red{client} maintains a $2D$ state space.}

  % \fignocaption{width = 0.50\textwidth}{figs/jupiter-illustration-client3-notations}
  \fignocaption{width = 0.60\textwidth}{figs/jupiter-illustration}

  \centerline{\large The \red{server} maintains $n \; (=3)$ $2D$ state spaces, one for each client.}
\end{frame}
%%%%%%%%%%%%%%%%%%%%

%%%%%%%%%%%%%%%%%%%%
\begin{frame}{}
  \begin{center}
    {\large \red{Global property} on all replica states specified by \blue{\wlspec{}}}

    \vspace{0.20cm}
    \fignocaption{width = 0.30\textwidth}{figs/mismatch}
    \vspace{0.20cm}

    {\large \red{Local view} each replica maintains in \blue{Jupiter}}
  \end{center}
\end{frame}
%%%%%%%%%%%%%%%%%%%%
