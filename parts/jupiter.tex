% file: parts/jupiter.tex

%%%%%%%%%%%%%%%%%%%%
\begin{frame}{}
  \centerline{\Huge \teal{Jupiter}}
\end{frame}
%%%%%%%%%%%%%%%%%%%%

%%%%%%%%%%%%%%%%%%%%
\begin{frame}{}
  \centerline{\large Jupiter adopts the \red{client-server} architecture~\ncite{Nichols:UIST95}:}

  \begin{center}
    \begin{minipage}{0.50\textwidth}
      % file: tikz/jupiter-cs-tikz.tex

\def\s{s}  % server
\def\b{bot} % literal string
\tikzset{seq/.style = {draw, circle, outer sep = 5pt, inner sep = 2pt, scale = #1, left}}

% send: the sender sends operation to the receiver
\newcommand{\send}[5]{% #1: sender; #2: receiver; #3: sender pos; #4: receiver pos; #5: seq. number;
  \draw[->]  ($(#1)!#3!(#1\b)$) node[seq = 0.40] {#5} to ($(#2)!#4!(#2\b)$) node[seq = 0.60] {$#5$};
}

% csend: client sends operation to the server
\newcommand{\csend}[6]{% #1: client; #2: client pos; #3: server pos; #4: seq. number; #5: client state; #6: server state
  \draw[->]  ($(#1)!#2!(#1\b)$) node[seq = 0.60, draw = none] (#4) {} to ($(\s)!#3!(\s\b)$) {};
}

% ssend: the server sends operation to client 
\newcommand{\ssend}[5]{% #1: client; #2: server pos; #3: client pos; #4: seq. number; #5: client state
  \draw[->, dashed]  ($(\s)!#2!(\s\b)$) to ($(#1)!#3!(#1\b)$) node[solid, seq = 0.60] {$#4$};
}

\newcommand{\ins}[2]{$\textcolor{blue}{\textsc{Ins}(#1,#2)}$}
\newcommand{\del}[2]{$\textcolor{blue}{\textsc{Del}(#2)}$}  % del(a,p): ignoring the element 'a'

\begin{tikzpicture}[
    timeline/.style = {very thick}, >=Stealth, 
    op/.style = {font = \small, above left = -0.20cm and -0.40cm of #1, sloped},
    replica/.style = {align = center}]
  \uncover<1->{
    \node[replica] (\s) {\textcolor{brown}{$S$}};
    \node[replica, right = of s] (c1) {$\textcolor{brown}{C_1}$};
    \node[replica, right = of c1] (c2) {$\textcolor{brown}{C_2}$};
    \node[replica, right = of c2] (c3) {$\textcolor{brown}{C_3}$};

    \foreach \r/\rbot in {s/sbot, c1/c1bot, c2/c2bot, c3/c3bot} {
	  \node[below = 5.0cm of \r] (\rbot) {};
	  \draw[timeline] (\r) to (\rbot);
    }
  }

  \uncover<2->{
    % Clients send messages (ordered by positions at server) to the server
    \csend{c1}{0.15}{0.20}{1}{x}{x}
    \csend{c1}{0.35}{0.40}{2}{}{}
    \csend{c2}{0.45}{0.55}{3}{ax}{a}
    \csend{c3}{0.50}{0.75}{4}{xb}{}
  }

  \uncover<3->{
    % Operations are totally ordered at the server.
    \foreach \num/\pos in {1/0.20, 2/0.40, 3/0.55, 4/0.75} {
      \node [seq = 0.80, fill = red!40] at ($(\s)!\pos!(\s\b)$) {$\num$};
    }
  }

  % \node (o1) [op = {1}, rotate = 7] {};
  % \node (o2) [op = {2}, rotate = 7] {};
  % \node (o3) [op = {3}, rotate = 8] {};
  % \node (o4) [op = {4}, rotate = 15] {};

  \uncover<4->{
    % The server sends messages (ordered by positions at server) to clients.
    \ssend{c2}{0.40}{0.65}{2}{a}
    \ssend{c3}{0.40}{0.70}{2}{b}

    \ssend{c2}{0.20}{0.30}{1}{x}
    \ssend{c3}{0.20}{0.25}{1}{x}


    \ssend{c1}{0.55}{0.60}{3}{a}
    \ssend{c3}{0.55}{0.90}{3}{}

    \ssend{c1}{0.75}{0.90}{4}{}
    \ssend{c2}{0.75}{0.90}{4}{}
  }
\end{tikzpicture}

    \end{minipage}
  \end{center}

  \vspace{-1.00cm}
  \uncover<2->{
    \begin{center}
      Operations are \red{totally ordered} at the server. \\[6pt]
      Client $\xrightarrow[]{\quad \text{FIFO} \quad}$ server $\xrightarrow[]{\quad \text{FIFO} \quad}$ other clients
    \end{center}
  }
\end{frame}
%%%%%%%%%%%%%%%%%%%%

%%%%%%%%%%%%%%%%%%%%
\begin{frame}{}
  \begin{center}
    {\large To achieve convergence, Jupiter uses \red{$2D$ state spaces}~\ncite{Xu:CSCW14}
    to manage how and when to perform \red{OTs}~\footnote{OT: Operational Transformation}~\ncite{Ellis:SIGMOD89}.}
  \end{center}

  \fignocaption{width = 0.32\textwidth}{figs/2d-statespace}

  \begin{center} 
    There can be \red{$\le 2$ edges} coming from the same node ({\textsc{\footnotesize Local}} or \textsc{\footnotesize Global}).
  \end{center}
\end{frame}
%%%%%%%%%%%%%%%%%%%%

%%%%%%%%%%%%%%%%%%%%
\begin{frame}{}
  \centerline{\large Each \red{client} maintains a $2D$ state space.}

  % \fignocaption{width = 0.50\textwidth}{figs/jupiter-illustration-client3-notations}
  \fignocaption{width = 0.60\textwidth}{figs/jupiter-illustration}

  \centerline{\large The \red{server} maintains $n \; (=3)$ $2D$ state spaces, one for each client.}
\end{frame}
%%%%%%%%%%%%%%%%%%%%

%%%%%%%%%%%%%%%%%%%%
\begin{frame}{}
  \begin{center}
    {\large \red{Global property} on all replica states specified by \blue{\wlspec{}}}

    \vspace{0.20cm}
    \fignocaption{width = 0.30\textwidth}{figs/mismatch}
    \vspace{0.20cm}

    {\large \red{Local view} each replica maintains in \blue{Jupiter}}
  \end{center}
\end{frame}
%%%%%%%%%%%%%%%%%%%%
