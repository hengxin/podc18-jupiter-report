% file: parts/cjupter.tex

%%%%%%%%%%%%%%%%%%%%
\begin{frame}{}
  \centerline{\Huge \teal{CJupiter (Compact Jupiter)}}
\end{frame}
%%%%%%%%%%%%%%%%%%%%

%%%%%%%%%%%%%%%%%%%%
\begin{frame}{}
  \begin{center}
    {\large CJupiter maintains an \red{$n$-ary ordered state space} for each replica.}
  \end{center}

  \fignocaption{width = 0.40\textwidth}{figs/cjupiter-allinone}

  \begin{center} 
    \pause
    There can be \red{more than two edges} coming from the same node. \\[5pt]
    Edges from the same node are \red{totally ordered} by associated operations.
  \end{center}
\end{frame}
%%%%%%%%%%%%%%%%%%%%

%%%%%%%%%%%%%%%%%%%%
\begin{frame}{}
  \begin{center}
    \begin{prop}[Compactness of CJupiter (Informal)]
      {\large At a high level, CJupiter maintains only \red{one} $n$-ary ordered state space.}
    \end{prop}

    % \resizebox{0.50\textwidth}{!}{\input{tikz/cjupiter-allinone-path}}
    \fignocaption{width = 0.40\textwidth}{figs/cjupiter-allinone-path}

    Each replica behavior corresponds to a \red{path} going through this state space.
  \end{center}
\end{frame}
%%%%%%%%%%%%%%%%%%%%

%%%%%%%%%%%%%%%%%%%%
\begin{frame}{}
  \begin{Theorem}[Equivalence of CJupiter and Jupiter]
    Under the same schedule, the behaviors of corresponding replicas in CJupiter and Jupiter are the same.
  \end{Theorem}

  \vspace{1.0cm}
  \centerline{\large From the perspectives of both the server and clients.}
\end{frame}
%%%%%%%%%%%%%%%%%%%%
