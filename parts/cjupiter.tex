% file: parts/cjupter.tex

%%%%%%%%%%%%%%%%%%%%
\begin{frame}{}
  \centerline{\Huge \teal{CJupiter (Compact Jupiter)}}
\end{frame}
%%%%%%%%%%%%%%%%%%%%

%%%%%%%%%%%%%%%%%%%%
\begin{frame}{}
  \begin{center}
    {\large CJupiter maintains an \red{$n$-ary ordered state space} for each replica.}
  \end{center}

  \fignocaption{width = 0.40\textwidth}{figs/cjupiter-allinone}

  \begin{center} 
    \pause
    There can be \red{more than two edges} coming from the same node. \\[5pt]
    Edges from the same node are \red{totally ordered} by associated operations.
  \end{center}
\end{frame}
%%%%%%%%%%%%%%%%%%%%

%%%%%%%%%%%%%%%%%%%%
\begin{frame}{}
  \begin{center}
    \begin{prop}[Compactness of CJupiter (Informal)]
      {\large At a high level, CJupiter maintains only \red{one} $n$-ary ordered state space.}
    \end{prop}

    % \resizebox{0.50\textwidth}{!}{% file: tikz/cjupiter-allinone-path.tex
% CSS lattice for jupiter-scheduling-podc16.tex

% default horizontal/vertical distance
\def\hdist{1.8}
\def\vdist{1.8}

\newcommand{\state}[2]{% #1: state label; #2: position
  \node (#1) [circle, inner sep = 0pt, minimum size = 10mm, text width = 10mm, align = center, draw, #2, font = \Large] {$#1$};
}

\tikzset{every lower node part/.style = {red}}
\newcommand{\statesplit}[3]{% #1: state upper label; #2: state lower label; #3: position
  \node (#1) [circle split, draw, minimum size = 6mm, text width = 10mm, align = center, #3, font = \Large]
  {
	$#1$
	\nodepart{lower}
	$#2$
  };
}

\newcommand{\transition}[4][]{% #2: start state; #3: end state; #4: transition label; #1: transition label position (optional)
  \draw[>=Stealth, ->] (#2) to node [rectangle, draw, above = 2pt, sloped, #1, font = \small] {#4} (#3);
}

\newcommand{\ins}[2]{$\textsc{Ins}(#1,#2)$}
\newcommand{\del}[2]{$\textsc{Del}(#1,#2)$}

\tikzset{node distance = \vdist and \hdist}
\tikzset{path/.style = {draw, rounded corners, ultra thick, #1}}

\begin{tikzpicture}
  \uncover<1->{
    \statesplit{0}{\epsilon}{}
    \statesplit{1}{x}{below = of 0}
    \transition{0}{1}{1: \ins{x}{0}}

    \statesplit{12}{\epsilon}{below left = of 1}
    \statesplit{13}{ax}{below right = of 1}
    % \statesplit{123}{a}{below = 2.1*\vdist of 1}
    \statesplit{123}{a}{below right = of 12}
    \transition{1}{12}{2: \del{x}{0}}
    \transition{1}{13}{3: \ins{a}{0}}
    \transition{12}{123}{3: \ins{a}{0}}
    \transition{13}{123}{2: \del{x}{1}}

    % \statesplit{14}{xb}{right = 1.5*\vdist of 13}
    % \transition{1}{14}{4: \ins{b}{1}}

    % \statesplit{124}{b}{right = 2.1*\hdist of 123}
    \statesplit{124}{b}{below right = of 13}
    \transition[near end]{12}{124}{4: \ins{b}{0}}
    
    % \statesplit{1234}{ba}{below = 2*\vdist of 13}
    \statesplit{1234}{ba}{below right = of 123}
    \transition{124}{1234}{3: \ins{a}{1}}
    \transition{123}{1234}{4: \ins{b}{0}}

    \statesplit{14}{xb}{above right = of 124}
    \transition{1}{14}{4: \ins{b}{1}}
    \transition{14}{124}{2: \del{x}{0}}
  }

  \uncover<2->{
    \path[path = {red}] ($(0.north) + (135:20pt)$) node[left] 
	  (sc1) {$S, C_1$} -- ++(-90:2.5*\vdist) -- ++(-135:2.5*\vdist) -- ++(-45:5.2*\vdist); % -- ++(-50:2.6*\vdist);
    \path[path = {blue}] ($(0.north) + (100:20pt)$) node[above = 0pt] 
	  (c2) {$C_2$} -- ++(-90:2.7*\vdist) -- ++(-45:2.5*\vdist) -- ++(-135:2.2*\vdist) -- ++(-45:2.4*\vdist);
    \path[path = {teal}] ($(0.north) + (60:15pt)$) node[right] 
	  (c3) {$C_3$} -- ++(-90:2.5*\vdist) -- ++(-18:5.5*\vdist) -- ++(-135:5.0*\vdist);
  }
\end{tikzpicture}}
    \fignocaption{width = 0.40\textwidth}{figs/cjupiter-allinone-path}

    Each replica behavior corresponds to a \red{path} going through this state space.
  \end{center}
\end{frame}
%%%%%%%%%%%%%%%%%%%%

%%%%%%%%%%%%%%%%%%%%
\begin{frame}{}
  \begin{Theorem}[Equivalence of CJupiter and Jupiter]
    Under the same schedule, the behaviors of corresponding replicas in CJupiter and Jupiter are the same.
  \end{Theorem}

  \vspace{1.0cm}
  \centerline{\large From the perspectives of both the server and clients.}
\end{frame}
%%%%%%%%%%%%%%%%%%%%
