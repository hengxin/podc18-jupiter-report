% file: parts/cjupter.tex

%%%%%%%%%%%%%%%%%%%%
\begin{frame}{}
  \centerline{\Huge \teal{CJupiter (Compact Jupiter)}}
\end{frame}
%%%%%%%%%%%%%%%%%%%%

%%%%%%%%%%%%%%%%%%%%
\begin{frame}{}
  \begin{center}
    {\large CJupiter maintains an \red{$n$-ary ordered state space} for each replica.}
  \end{center}

  \fignocaption{width = 0.50\textwidth}{figs/cjupiter-allinone}
\end{frame}
%%%%%%%%%%%%%%%%%%%%

%%%%%%%%%%%%%%%%%%%%
\begin{frame}{}
  \begin{center}
    \begin{prop}
      {\large At a high level, CJupiter maintains only \red{one} $n$-ary ordered state space.}
    \end{prop}

    \resizebox{0.50\textwidth}{!}{\input{tikz/cjupiter-allinone-path}}
    % \fignocaption{width = 0.45\textwidth}{figs/cjupiter-allinone-path}

    \uncover<2->{Each replica behavior corresponds to a \red{path} going through this state space.}
  \end{center}
\end{frame}
%%%%%%%%%%%%%%%%%%%%

%%%%%%%%%%%%%%%%%%%%
\begin{frame}{}
  \begin{Theorem}[Equivalence]
    equiv
  \end{Theorem}
\end{frame}
%%%%%%%%%%%%%%%%%%%%

%%%%%%%%%%%%%%%%%%%%
\begin{frame}{}
\end{frame}
%%%%%%%%%%%%%%%%%%%%
