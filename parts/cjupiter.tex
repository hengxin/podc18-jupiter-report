% file: parts/cjupter.tex

%%%%%%%%%%%%%%%%%%%%
\begin{frame}{}
  \centerline{\Huge \teal{CJupiter (Compact Jupiter)}}
\end{frame}
%%%%%%%%%%%%%%%%%%%%

%%%%%%%%%%%%%%%%%%%%
\begin{frame}{}
  \begin{center}
    {\large CJupiter maintains an \red{$n$-ary ordered state space} for each replica.}
  \end{center}

  \fignocaption{width = 0.40\textwidth}{figs/cjupiter-allinone}

  \begin{center} 
    \pause
    There can be more than two edges coming from the same node. \\[6pt]
    \pause
    Edges from the same node are \red{totally ordered} by associated operations.
  \end{center}
\end{frame}
%%%%%%%%%%%%%%%%%%%%

%%%%%%%%%%%%%%%%%%%%
\begin{frame}{}
  \begin{center}
    \begin{prop}[$n + 1 \to 1$ (Informal)]
      {\large At a high level, CJupiter maintains only \red{one} $n$-ary ordered state space.}
    \end{prop}

    \resizebox{0.50\textwidth}{!}{\input{tikz/cjupiter-allinone-path}}

    \uncover<2->{Each replica behavior corresponds to a \red{path} going through this state space.}
  \end{center}
\end{frame}
%%%%%%%%%%%%%%%%%%%%

%%%%%%%%%%%%%%%%%%%%
\begin{frame}{}
  \begin{Theorem}[Equivalence]
    Under the same schedule, the behaviors of corresponding replicas in CJupiter and Jupiter are the same.
  \end{Theorem}

  \pause
  \vspace{0.30cm}
  \centerline{\large At the server side:}
  \begin{prop}[$n \leftrightarrow 1$ (Informal)]
    The single $n$-ary ordered state space at the server side in CJupiter 
    is a \red{compact representation} of $n$ $2D$ state spaces at the server side in Jupiter.
  \end{prop}

  \vspace{0.30cm}
  \centerline{\large At the client side:}
  \begin{prop}[$1 \leftrightarrow 1$ (Informal)]
    Jupiter is \red{slightly optimized in implementation} at clients by eliminating redundant OTs than CJupiter.
  \end{prop}
\end{frame}
%%%%%%%%%%%%%%%%%%%%
